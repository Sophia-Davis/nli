
\documentclass{article}
% \usepackage[utf8]{inputenc}

\title{Progress Log}
\maketitle
\begin{document}

\section{Sunday, 30 October 2016: Initial Brainstorming}

	Today, began thinking of strategies to implement to figure out the open problem of native speaker identification. 
	To start, I read the report on the 2013 NLI competition using ETS data. 
	Decided to use an SVM, as it was the method with highest rates of success in competition. 
	
	Considered features to use in SVM Among them 
	\begin{enumerate}
		\item Word Unigrams. Most likely, best to use spellcorrected features as unigrams, as the dataset is not large enough for misspellings to be systematic.  
			
		\item Compute how words are misspelled (compute levenstein delta from levenstein distance)
		\item Simple statistics such as: number of words per sentence, number of characters per word, number of words per sentence, percent of capitalized characters/words
		\item Character bigrams			
			
	\end{enumerate}

\section{Saturday,  26 November 2016: Word Unigrams, SVC}

First, got dictionary of all word unigrams. accuracy of 66\% with linear SVC.\\
Accuracy with unigrams:
.65727273  0.66090909  0.6666666

Later, combined array of word unigrams with that of character bigrams. Results were dissapointing: accuracy not higher than with just word unigrams. 
accuracy with unigrams and char bigrams: 0.64666667  0.66030303  0.65606061]

I decided to change model from linear SVC (less advanced) to SVC. 
This was not much of a change; scikit allows you to ``plug and chug''.  
\end{document}
